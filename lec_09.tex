\lecture{9}{March 23, 2021}{Ch. 27 (2/2), Ch. 28}

\paragraph{Nonincome Determinants of Consumption and Saving}

\begin{enumerate}[label = (\alph*)]
	\item \textbf{Wealth.} Because one has more wealth, the weaker the incentive to save. Thus, the larger the consumption and the larger the savings.

	\item \textbf{Real interest rate (related to wealth determinant).} If there is a decrease in interest rates, consumption is going to increase and savings are going to drop. Changes in interest rates can change the real value or the purchasing power of certain types of wealth.

	\item \textbf{(Consumer) Expectations.} If you expect price to increase or a shortage to occour in the future or your income increase in the future, your consumption increases and savings drop (now).

	\item \textbf{Consumer Indebtedness.} If consumers' debt is high, then households are going to decrease their spending to decrease their indebtedness.

	\item \textbf{Taxation.} An increase in taxes is going to decrease both consumption and savings. Taxes are paid partly at the expense of consumption and partly at the expente of savings.

\end{enumerate}

\subsection{Investment}

\subsubsection{Determinants of Investment}

\begin{enumerate}[label = (\arabic*)]
	\item \textbf{Expected Rate of Return.} Business and individuals invest with respect to their expectations. The more profit they make, the more incentive to invest. 

		Expected Rate of Return = Profit / Investment Cost.

	\item \textbf{Real Interest Rate.} The cost of borrowing the necessary money.

		If the expected rate of return is equal to or greater than the real interest rate, then one should invest.

\end{enumerate}

\subsubsection{Investment Demand Curve}

\begin{enumerate}
	\item \textbf{Introduction.}
	\item \textbf{Inverse Relationship.} (see figure 9--5)

		As the real interest rate drops, the investment spendings are going up.

	\item What does the curve tell us?
	\item (see figure 9--6)
\end{enumerate}

\newpage
\section{Chapter 28: The Aggregate Expenditures Model}

\subsection{Expenditures--Output Model}

\subsubsection{Graphic Analysis}

\paragraph{Use 45 degree line} The value of what is measured on the horizontal and vertical axis are the equal.

\paragraph{Equilibrium Condition Level} $C + I_g = \text{Real GDP}$

\paragraph{Plot old Consumption and DI graph with Investment} Add \$20 Billion of Investment to Consumption schedule. (see table 9--4 and figure 9--9).

\subsubsection{Leakages--Injections Approach}

\paragraph{Define.} It is the determination of Equilibrium Real GDP by finding where the leakages equal the injections.

\begin{enumerate}[label = (\alph*)]
	\item Means Saving (leakage): Consumptions will fall short in relation to GDP.
	\item Means Investment (injections): Supplements consumption with capital and investment goods. 
\end{enumerate}

\paragraph{Conlude}
\begin{enumerate}[label = \textbullet]
	\item If $I > S$, then $C + I_g > GDP$.
	\item If $I < S$, then $C + I_g < GDP$.
	\item If $I = S$, then Equilibrium GDP.
\end{enumerate}

\subsubsection{The Multiplier}
\paragraph{Changes in $I$ or $C$ and $S$ will result in changes in Equilibrium GDP.}
* Choose Investment because it is more volatile.

\paragraph{If the expected rate of return increases or real interest rate falls,} what happens to investment spending?

\paragraph{Multiplier Effect}
\begin{enumerate}[label = \textbullet]
	\item Is the effect upon equilibrium $GDP$ due to a change in aggregate expenditures.
	\item Measure the Multiplier:
		\[
			\text{Multiplier} = \frac{\Delta \text{ in Real GDP}}{\Delta \text{ initial change in expenditure}}
		\]
\end{enumerate}

\paragraph{Example}

\paragraph{Also the multiplier can be calculated as} \[
	\text{Multiplier} = \frac{1}{\text{MPS}} = \frac{1}{1 - \text{MPC}} 
\]
