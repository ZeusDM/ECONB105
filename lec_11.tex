\lecture{11}{April 02, 2021}{Ch. 30 (2/2), Ch. 31(1/2)}

\subsection{Non Discretionary Fiscal Policy}

\paragraph{Definition} Increases and decreases in net taxes which occur without congressional action/intervention.
This non discretionary fiscal policy also brings stability to economic system but automatically.

\subparagraph{Net Taxes}\
\[
	\text{net taxes} = T - \text{transfers and subsidies}.
\]

\begin{enumerate}[label = (\alph*)]
	\item \textbf{Net Taxes vary directly with GDP.} (positive correlation)
	\item \textbf{Progressive Tax System:} As GDP increases, income goes up, and individuals are pushed into a higher tax bracket. Thus, tax revenue will increase (especially for inelastic products).
\end{enumerate}

\subparagraph{Built--in--Stabilizers} (counter-cyclical)
\begin{enumerate}[label = \textbullet]
	\item \textbf{Definition of a Built-in-Stabilizer:} Anything which tends to offset inflation and/or unemployment.

	In a recession, anything that increases the government deficit or decreases the surplus is a built-in-stabilizer. In times of inflation, anything that decreases the government deficit or increases the surplus in a built-in-stabilizer.

	\item \textbf{Tax Revenues are a stabilizer:} leakage
		\begin{enumerate}[label = \textbullet]
			\item if leakages increase: the overall purchasing power decrease. Thus, it would offset inflation.
			\item if leakages decrease: the overall purchasing power increase. Thus, it would offset recession.
		\end{enumerate}

	\item[(e.g.)] Unemployment compensation and welfare are others built-in-stabilizers.
\end{enumerate}

\subsection{Budgets}

\begin{enumerate}[label = \textbullet]
	\item When $G = T$, we have a balanced budget.
	\item When $G > T$, we have a deficit.
		Now, when $G > T$ is OK but $G = T + \text{borrowing}$ (to finance government spending).
\end{enumerate}

\subsection{Deficits and Debt}

\paragraph{Definition of Budged Deficit:} It is the amount that $G$ exceeds tax revenues from one year. 

\paragraph{Definition of Public Debt:} It is the total accumulation of deficits and surpluses that have occurred through time. (Does not include state and local governments.)

\subsection{Budget Philosophies}

\paragraph{Is a balance budget desirable?}\ 

\begin{enumerate}[label = \textbullet]
	\item Prior to 1930's, it was accepted to be.
	\item After 1930's, it was not desirable, because it was considered pro-cyclical.
		\begin{enumerate}[label = (\alph*)]
			\item (Pro-cyclical) If it moves with the bussiness cycle, it rules out fiscal activity which is a counter-cyclical stabilizer force. Therefore, it intensifies the business cycle.
			\item[(e.g.)] During a recession, $GDP$ decreases; employment decreases; taxes revenues decrease. To balance the budget, the government would decrease government spending/increase the taxes. \emph{But, this is a contractionary policy!} It would make the recession worse.
		\end{enumerate}
\end{enumerate}

\paragraph{Balanced Budget Alternatives}\ 

\begin{enumerate}[label = \textbullet]
	\item \textbf{Cyclicaly Balanced Budget}\ 

		Instead of balancing anually, the idea is to balance the budged during the business cycle.  A problem is that the stages of business cycle can differ in magnetude and duration. A problem is that if the recession is long and the recovery was short, the government will have a debt.

	\item \textbf{Functional Finance}\ 

		The idea is to balance the economy, not the budget. The government will provide a full employment non inflationary state regardless of the debt.  A problem is that a lot of times this means spending more, and growing budged deficit.
	
\end{enumerate}

\section{Chapter 31: Money, Banking and Financial Institutions}

\subsection{Functions of Money}

\paragraph{What is money?}\ 

\begin{enumerate}[label = \textbullet]
	\item \textbf{Medium of Exchange:} used to buy and sell goods.

	\item \textbf{Unit of Account:} measures the relative worth of goods and services. It is a common denominator to compare goods and services.

	\item \textbf{Store of Value:} money is the most convenient form of store wealth. It is the more liquid asset.
\end{enumerate}

\subsection{The Supply of Money}

\paragraph{Money Definition} $M_1$

\begin{align*}
	M_1& = \text{Currency}\\& + \text{Transactions Deposits}\\& + \text{Travelers Checks}
\end{align*}

\begin{enumerate}[label = (\arabic*)]
	\item \textbf{Currency.}

		\begin{enumerate}[label = \textbullet]
			\item \textit{Token money (coins).} The intrinsic value (the value of the coin itself) is less than its face value.

			\item \textit{Paper money (federal reserve note).} Issued by federal reserve bank.
		\end{enumerate}
	\item \textbf{Transactions deposits.} Debitable (debit card) or checkable (checkings account). Offer safety and convenience. It is the most common form of money.
\end{enumerate}

\paragraph{Money Definition} $M_2$

\begin{align*}
	M_2 = M_1 &+ \text{savings deposits}\\
	& + \text{small (less than \$100,000) time deposit}\\
	& + \text{money market deposit accounts}\\
	& + \text{money market mutual fund balances}.
\end{align*}

\begin{enumerate}[label = (\arabic*)]
	\item \textbf{Near Monies.}
		They are highly liquid financial assets, i.e., they are easily converted into currency and transaction deposits.
\end{enumerate}

