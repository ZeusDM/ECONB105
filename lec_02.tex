\lecture{2}{February 19, 2021}{Ch. 1 (2/2), Ch. 2, Ch. 3 (1/3)}

\subsection{Changing the assumptions}

\paragraph{Unemployment of resources} Represented by datapoint $U$ on \cref{fig:breadvsrobots}.

\paragraph{Technology increasing} Represented by a shift of the curve to up and right.

\subsection{Extra}

\paragraph{Arthur Laffer} Low taxes incentivize people to work more. Thus, it makes economies grow.

\newpage
\section{Chapter 2: The Market System and Circular Flow}

\subsection{Five Fundamental Questions}

\paragraph{What goods and services will be produced?} 

\paragraph{How will the goods and services be produced?} What combination of resources and technologies will be used to produce goods and services? How will the production be organized?

\paragraph{Who will recieve the output?} How should total output of goods be shared? Suggestions: based on need; based on contribution to product.

\paragraph{How will the system adapt to change?} Can the economic system change fast enough to remain efficient? This implies a reallocation of resources, since consumer taste, resources and technology changes.

\paragraph{How will the system promote progress?} How do we get output increase? How to get economic growth? This means the standard of living goes up. Technological improvements and capital accumulation will promote this.

\subsection{The Economic Systems}

\begin{figure}[ht]
    \centering
	\incfig[.8]{communism-capitalism-spectrum}
    \caption{Communism--Capitalism Spectrum}
    \label{fig:communism-capitalism-spectrum}
\end{figure}

\paragraph{Industrial Advanced Economies}

\paragraph{Pure Capitalism} Private ownership of resources. Direct economic activity. Free market. There's no need for any government intervention.

\paragraph{The Command Economy: Communism} Public ownership of most resources. Economic decisions were made by a central economic planning board. 

\paragraph{Mixed Systems} Fall between Communism and Pure Capitalism. Examples:

\begin{enumerate}[label = \textbullet]
	\item \textbf{U.S.}\\There is some government intervention and ownership.\\[-1ex]

	\item \textbf{Authoritarian Capitalism, e.g. Nazi Germany} \\ Privately owned resources. Heavy government control over the markets.\\[-1ex]

	\item \textbf{Market socialism, e.g. Serbia} \\ There is public ownership of resources, but also a partial free market.\\[-1ex]

	\item \textbf{Traditional/Costumary Economy, e.g. some Middle East countries} \\ Property ownership and government intervention are basted on customs. Religions and cultural values dictate economic activity.
\end{enumerate}

\newpage
\section{Chapter 3: Demand, Supply and Market Equilibrium}

\subsection{Price Determination}

\paragraph{Introduction} Why does the cost of ––– is –––?

\paragraph{Price} Price is a measure of a product's value.

\paragraph{Objective Value} Cost of production value. Based on supply.

\paragraph{Subjective Value} Based on individual preferences. Based on demand.
