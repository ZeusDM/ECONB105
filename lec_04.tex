\lecture{4}{February 26, 2021}{Ch. 3 (3/3)}
\paragraph{Determinants of Supply}
\begin{enumerate}[label = \textbullet]
	\item Costs of production: the cheaper it is to produce, the greater will be the supply. This can be divided into:
		\begin{enumerate}[label = ---]
			\item Resource prices: the price of resources used to produce a product (negative correlation).
			\item Technology: (positive correlation).
			\item Taxes and subsidies: taxes (negative correlation) and subsidies (positive correlation).
		\end{enumerate}
	\item Prices of other goods: substitution in prodution (negative correlation).
	\item Producer Expectations: future prices, \dots
	\item Number of sellers in the market: the more suppliers, the greater the supply.
\end{enumerate}
\paragraph{Changes in Supply} Movement of the entire supply curve. See \cref{fig:changes_in_supply}.

\begin{figure}[ht]
	\centering
	\incfig[.4]{changes_in_supply}
	\caption{Changes in supply}
	\label{fig:changes_in_supply}
\end{figure}

\subsection{Market Equilibrium}
\paragraph{Equilibrium of Supply and Demand} Price is determined where quantity supplied equals quantity demanded.
$P_E$ is the equilibrium price or market-clearing price, where the intentions of buyers and sellers match.
See \cref{fig:equilibrium_of_supply_and_demand}.

\begin{figure}[ht]
	\centering
	\incfig[.4]{equilibrium_of_supply_and_demand}
	\caption{Equilibrium of Supply and Demand}
	\label{fig:equilibrium_of_supply_and_demand}
\end{figure}

\paragraph{Disequilibrium of Supply and Demand}\ 

\begin{enumerate}[label = \textbullet]

	\item If the price is above the equilibrium price, we have an example of excess supply, or surplus.
		Surpluses drive prices down.
		See \cref{fig:excess_supply}.

	\begin{figure}[ht]
		\centering
		\incfig[.4]{excess_supply}
		\caption{Excess supply}
		\label{fig:excess_supply}
	\end{figure}

	\item If the price is below the equilibrium price, we have an example of excess demand, or shortage.
		Shortages drive prices up.
		See \cref{fig:excess_demand}.

	\begin{figure}[ht]
		\centering
		\incfig[.4]{excess_demand}
		\caption{Excess demand}
		\label{fig:excess_demand}
	\end{figure}

\end{enumerate}

\paragraph{Market Equilibrium Application} INSERT TABLE 3-8

\paragraph{Market Desiquilibrium Application}

\begin{enumerate}[label = \textbf{(\Roman*)}]
	\item \textbf{Labor Market}

		\begin{enumerate}[label = ---]
			\item $Q_a$: labor demand with minimum wage.
			\item $Q_e$: employment without minimum wage.
			\item $Q_b$: labor supply with minimum wage (laborers who want to work at minimum wage)
		\end{enumerate}

		INSERT HANDMADE GRAPH

	\item \textbf{Price Ceilings:} Rent control (lessors cannot raise the price).

		There is excess demand.

		INSERT HANDMADE GRAPH
\end{enumerate}

\subsection{Changes in Supply and Demmand}

INSERT FIGURE 3.7

\begin{tabular}{lll}
	$D$ increase: &$P \uparrow$ 	&$Q \uparrow$	\\
	$D$ decrease: &$P \downarrow$	&$Q \downarrow$	\\
	$S$ increase: &$P \downarrow$	&$Q \uparrow$	\\
	$S$ decrease: &$P \uparrow$ 	&$Q \downarrow$
\end{tabular}
