\newpage
\section{Chapter 30: Fiscal Policy, Deficits and Debt}
\lecture{10}{March 26, 2021}{Ch. 30 (1/2)}

\subsection{Equilibrium vs. Full Employment GDP}

\paragraph{Introduction}\ 

\paragraph{Recessionary and Inflationary Gaps}\ 

\begin{enumerate}[label = \textbf{(\arabic*)}]
	\item \textbf{Recessionary Gap:} deficiency of spending.

		Recessionary gap is the amount that aggregate expenditures fall short of the non inflationary full employment GDP.

		INSERT HANDMADE GRAPH

	\item \textbf{Inflationary Gap:} excess of spending.

		Inflationary gap is the amount by which aggregate expenditures exceed the full employment non inflationary  GDP.

		INSERT HANDMADE GRAPH

\end{enumerate}

\subsection{Government Spending ($G$)}

\paragraph{How do we adjust for Gaps?}\ 

\paragraph{Introduction}\ 

\begin{enumerate}[label = \textbullet]
	\item \textbf{Fiscal Policy:} government budget policy ($G$ and taxes)
	\item \textbf{Monetary Policy:} FED (Federal Reserve System) (changes in the money supply)
\end{enumerate}

\paragraph{Fiscal Policy} Changes in the Public Sector.

If taxes increase by \$20 bilion, then disposable income is going to decrease by \$20 bilion. Using $MPC = .75$ and $MPS = .25$, the consumption decreases by \$15 bilion and savings decreasse by \$5 bilion.

\subsection{Discretionary Fiscal Policy}

\paragraph{Definition} Deliberate changes in $T$ and $G$ by congress for the purpose of achieve the full employment non inflationary GDP.

\paragraph{Equation} $Y = C + I + G$, where $Y$ stands for real GDP.

\begin{enumerate}[label = \textbf{(\arabic*)}]
	\item \textbf{Recessionary Gap:} where aggregate expenditures fall short of the full employment, non inflationary $Y$.

	\item \textbf{Inflationary Gap:} where expenditures exceed the full employment, non inflationary $Y$.

	\item How much increase or decrease $G$ or $T$?

		\[
			\Delta Y = k \times \Delta G,
		\]
		where $k = $ multiplier.
\end{enumerate}

\subsection{Stabilizing the Economy}

\paragraph{In times of a recession:} Expansionary Fiscal Policy is used.

\subparagraph{Means:}
\begin{enumerate}[label = (\alph*)]
	\item increase government expending and/or;
	\item decrease taxes.
\end{enumerate}

\paragraph{In times of inflation:} Contractionary Fiscal Policy is used.

\subparagraph{Means:}
\begin{enumerate}[label = (\alph*)]
	\item decrease government expending and/or;
	\item increase taxes.
\end{enumerate}

