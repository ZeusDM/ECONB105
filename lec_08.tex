\newpage
\section{Chapter 27: Basic Macroeconomic Relationships}
\lecture{8}{March 19, 2021}{Ch. 27 (1/2)}

\subsection{Classical Theory}

\subsubsection{``Laissey-Faire'' Economy}

	\paragraph{Capitalism can self adjust}\ 

	\paragraph{Real Wage of Labor} $=$ Money Wage $/$ Price Level

	\paragraph{Self Adjusting Mechanism}

	If the economy is in a recession, the price level is going down. Therefore, real wages go up. So, it is harder for employees to afford the wages; thus unemployment rises. Unemployed people now accept lower money wages. Real wages drop. Bussiness are able to hire more individuals. Unemployment goes back up.

\subsection{Keynesian Theory}

\subsubsection{Introduction} 1936, The General Theory
	
	\paragraph{Real Wage of Labor} $=$ Money Wage $/$ Price Level

	However, money wages are inflexible downward because of labor unions.
	Capitalism is not a self adjusting economy.

\subsection{The Kenesian Employment Theory}

\paragraph{What is the model?}

$C + I + G + X_n =$ Income generated by the productions and services

\subsubsection{Analytical Tools}

\paragraph{Assumptions:} no government consumption; only household saving $\therefore$ Gross Domestic Product = Disposable Income; constant price level.

\paragraph{Review the relationship of DI, C, and S:}
$
	DI = C  + S
$

\subsubsection{Consumption and Saving}

\paragraph{Consumption Schedule/Curve} Household spend a large porpotion of a small disposible income than a large disposable income.

\paragraph{Saving Schedule} Household save a smaller porpotion of a small disposible income than a large disposable income.

Therefore, as DI increases, households spend less and save more!

\paragraph{Application of schedules and 45\% line:}

\paragraph{Break Even Income:}

\begin{enumerate}[label = (\alph*)]
	\item Break even invome is the level at which the households consume the entire income, i.e., when DI = C.

	\item Dissaving/Saving.
\end{enumerate}

\subsubsection{Average and Marginal Propensities}

\paragraph{Average propensity to consume (APC):} Fraction of disposible income that household spend on goods and services.

\[
	APC = \frac{\text{Consumption}}{\text{Income}}.
\]

\paragraph{Average propensity to save (APS):} Fraction of disposible income that household save.

\[
	APS = \frac{\text{Saving}}{\text{Income}}.
\]

\paragraph{Marginal propensity to consume (MPC):} The rate of change of consumption with respect to disposable income.

\[
	MPC = \frac{\Delta \text{ in Consumption}}{\Delta \text{ in Income}}
\]

\paragraph{Marginal propensity to save (MPS):} The rate of change of savings with respect to disposable income.


\[
	MPS = \frac{\Delta \text{ in Saving}}{\Delta \text{ in Income}}
\]

For this course, we'll consider $MPC = .75$ and $MPS = .25$, unless otherwise stated.

