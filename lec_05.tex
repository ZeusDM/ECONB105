\newpage
\section{The Elasticity of Supply and Demand}
\lecture{5}{March 5, 2021}{Ch. 4, Ch. 24 (1/2)}

\subsection{Price Elasticity of Demand}

\begin{enumerate}[label = \textbf{(\Alph*)}, itemsep = 1em]

	\item Introduction
	\item Define

		\paragraph{Price Elastic} Responsive to price changes. In other words, modest price changes result in consideral change in quantity purchased. 
		\paragraph{Price Inelastic} Irresponsive to price changes. In other words, substantial price changes result in mostest change in quantity purchased. 

	\item Elasticity of Demand Formula

		\[ E_d = \dfrac{\%\Delta Q_d}{ \%\Delta P} = \dfrac{\dfrac{\Delta Q}{(Q_1 + Q_2)/2}}{\dfrac{\Delta P}{(P_1 + P_2)/2}} \]

		\begin{enumerate}[label = \textbf{(\arabic*)}, itemsep = 1ex]
			\item Rules
				\begin{enumerate}[label = ---, itemsep = 1ex]
					\item When $E_d > 1$ $\longrightarrow$ Elastic
					\item When $E_d < 1$ $\longrightarrow$ Inelastic	
					\item When $E_d = 1$ $\longrightarrow$ Unit Elasticity
				\end{enumerate}

			\item Qualifications
				\begin{enumerate}[label = \textbf{(\alph*)}, itemsep = 1ex]
					\item Ignore Minus Sign: $\Delta Q$ and $\Delta P$ are taken in absolute values.
					\item Midpoint Formula: The reference for the porcentages is the midpoint.
				\end{enumerate}
		\end{enumerate}

	\item Application of Formula, e.g. Home Depot has a paint sale.

		\begin{enumerate}[label = \textbf{(\arabic*)}, itemsep = 1ex]
			\item Interior Wall Paint

				\begin{multicols}{2}
					\begin{center}
						\begin{tabular}{rcc}\toprule
											& $P$ 		& $Q_d$			\\ \midrule
							Original Price 	& $\$16$ 	& $50$ gals		\\ \midrule
							Sale Price 		& $\$12$ 	& $100$ gals	\\ \bottomrule
						\end{tabular}
					\end{center}

					\emph{Find the Elasticity of Demand:}
					\[ E_d = \dfrac{\frac{50}{75}}{\frac{4}{14}}= \dfrac{7}{3} \approx 2.3 .  \]

					Thus, it is elastic.

				\end{multicols}
				

			\item Porch Floor Paint

				\begin{multicols}{2}
					\begin{center}
						\begin{tabular}{rcc}\toprule
											& $P$ 		& $Q_d$			\\ \midrule
							Original Price 	& $\$16$ 	& $20$ gals		\\ \midrule
							Sale Price 		& $\$12$ 	& $22$ gals		\\ \bottomrule
						\end{tabular}
					\end{center}
					
					\emph{Find the Elasticity of Demand:}
					\[ E_d = \dfrac{\frac{2}{21}}{\frac{4}{14}} = \frac{1}{3} \approx 0.3.  \]

					Thus, it is inelastic.
				\end{multicols}

			\item The Determinants of Price Elasticity of Demand

				\begin{enumerate}[label = \textbullet, itemsep = 1ex, topsep = 1ex]
					\item \textbf{Substutability:} the larger the number of good substitute products, the greater the elasticity of demand, i.e., the greater price sensitivity. E.g. lower price cars are sensitive to price; in contrast, heroin is inelastic since there are no real substitutes.
					\item \textbf{Proportion of income:} the more significant a purchase (i.e., the more money alloted to it) the greater the elasticity of demand.
					\item \textbf{Luxuries vs. necessities:} The demand for necessities tends to be inelastic, while the demand for luxuries tends to be elastic. E.g., insulin is inelastic since it is a necessity; Caribe vacations are elastic.
					\item \textbf{Time:} The demand for a product tends to be more elastic the longer the time period. When the price of a product goes up, it takes time to experement with other products to see if they are acceptable as a substitue. With time, we will find other acceptable products.

						For example, as the price of Expresso goes up, given some time, you could switch your taste to Regular Coffee.
				\end{enumerate}
		\end{enumerate}

\end{enumerate}

\subsection{The Elasticity of Supply}
\begin{enumerate}[label = \textbf{(\Alph*)}, itemsep = 1em]
	\item Introduction
	\item Definition. If producers are sensitive to price changes of what they are producing, then supply is elastic. If the producers are insensitive to price changes, then supply is inelastic.
	\item Supply Elasticity Formula

		\[ E_s = \dfrac{\%\Delta Q_s}{ \%\Delta P} = \dfrac{\dfrac{\Delta Q}{(Q_1 + Q_2)/2}}{\dfrac{\Delta P}{(P_1 + P_2)/2}}\]

		\begin{enumerate}[label = \textbf{(\arabic*)}, itemsep = 1ex]
			\item Rules
				\begin{enumerate}[label = ---, itemsep = 1ex]
					\item When $E_s > 1$ $\longrightarrow$ supply is elastic
					\item When $E_s < 1$ $\longrightarrow$ supply is inelastic	
					\item When $E_s = 1$ $\longrightarrow$ unit elasticity
				\end{enumerate}
		\end{enumerate}

			\item The Determinants of Price Elasticity of Supply
				\begin{enumerate}[label = \textbullet, itemsep = 1ex, topsep = 1ex]
					\item \textbf{Time:} The time a supplier/producer has to respond to a given change in the product price. The longer the amount of time a producer has to adjust to a price change, the greater the supply elasticity, i.e., the output response. The more time, the greater the ability of a producer to shift resources, which increases the elasticity of supply.
				\end{enumerate}
\end{enumerate}

\subsection{Supply and Demand Elasticity Appications}
\begin{enumerate}[label = \textbf{(\Alph*)}, itemsep = 1em, parsep = 1ex]
	\item \textbf{Tax on Cigarettes: Inelastic Demand.} When price increases due to a tax, demand is insensitive to the price change. Therefore, the quantity demanded decreases very little.

		The supply curve will decrease, but the loss of sale to the producer is small, because the demand is inelastic.

		The amount of impact the tax has on the sale of cigarettes is going to depend on the elasticity of the demand curve.

		INSERT HANDMADE GRAPH 
	\item \textbf{What if Cigarettes are Elastic?} When price encreases due to a tax, demand is sensitive to the price change. Therefore, the quantity demanded will decrease substantially.

		INSERT HANDMADE GRAPH

\end{enumerate}

\subsection{Income Elasticity of Demand}
\begin{enumerate}[label = \textbf{(\Alph*)}, itemsep = 1em]
	\item Definition: measures the responsiveness of consumer purchases to income changes.

		\[ E_i = \dfrac{\%\Delta \text{ in quantity demanded}}{ \%\Delta \text{ in income}} = \dfrac{\dfrac{\Delta Q}{(Q_1 + Q_2)/2}}{\dfrac{\Delta I}{(I_1 + I_2)/2}}\]

\end{enumerate}
\begin{enumerate}[label = \textbf{(\arabic*)}, itemsep = 1em]
	\item Normal Goods: positive elasticity. More normal goods are demanded when income goes up.

	\item Inferior Goods: negative elasticity. 
	
\end{enumerate}

\newpage
\section{Chapter 24}

\subsection{Measuring the Economy's Performance}



\begin{enumerate}[label = (\Alph*)]
	\item \textbf{Gross Domestic Product} is defined as the total market value of all final goods and services produced withing the nation boundaries in one year.

		\begin{enumerate}[label = ---]
			\item Final goods and services are goods and services purchased for final use. Not for resale and not for further processing.
			\item Intermediate goods and services are goods ans services purchased for resale or further processing.
		\end{enumerate}

	\item[\textbullet] Problems calculating GDP:

		\begin{enumerate}[label = (\arabic*)]
			\item Multiple Counting.

				E.g. steel and automobiles. 

				Solution:

			\item Money vs. Real GDP.

				Deflation and inflation changes the value of money.

				GDP is a sales figure. Price $\times$ quantity. Price is the problem, not the quantity.

				We need to account for inflation and deflation.

				Nominal GDP is not adjusted for price changes. Real GDP is adjusted for price changes.
		\end{enumerate}

	\item Adjusting for Nominal GDP

	\item Formulas and Application

		\begin{enumerate}[label = (\alph*)]
			\item $\text{price index} = \dfrac{\text{price of a market basket in any specific year}}{\text{price of the same market basket in base year}}$.

				Some examples are Consumer Price Index (CPI), Producer Price Index (PPI), GDP Deflator.

				INSERT CPI GRAPH, FIGURE 8-6.

			\item $\frac{\text{nominal GDP}}{\text{price index}} = \text{real GDP}$

				\begin{table}
					\begin{tabular}{rlll}
						(trillions) & Nominal & Price Index & Real GDP \\
						$2012$ & $\$ 16.197$ & $100.0 \%$ & \\ 
						$2017$ & $\$ 19.519$ & $107.19 \%$\\
						$2018$ \\
					\end{tabular}
				\end{table}
		\end{enumerate}

\end{enumerate}
