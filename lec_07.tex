\lecture{7}{March 12, 2021}{Ch. 26 (2/2)}

\paragraph{Two Types of Inflation}

\begin{enumerate}[label = (\arabic*)]
	\item \textbf{Demand Pull:} inflation which is the result of an increase in aggregate demand. 

		This is when the aggregate expenditures begin to approach the production capacity of the economy. If demand increases and supply is constant because they can't produce anymore, the people who want the product will bid higher.

		Coincides with high level of employment and high level of output.

	\item \textbf{Cost Push Inflation:} inflation which results in an decrease of aggregate demand.

		This is when output and employment are on decline, when average prices are going up. Rising prices are a result of labor unions and large coorperations. If wages go up, then the production cost rises, and the coorperations rise the price for the consumers.
	
\end{enumerate}

\subsection{Redistributive Effects of Inflation}

\paragraph{Introduction} Inflation can redistribute income (paycheck) and your wealth (accumulated equity, what you own).

If you apply Real GDP formula to income:

\[
	\text{Real Income} = \frac{\text{Nominal Income}}{\text{Price Index}}
\]

\paragraph{What is the difference between nominal income and real income?}
Nominal income is the number of dollars received in wages, rent, interest and profit. Real income is defined as the amount of goods and services one's nominal income will buy.

Using the same formula:

\[
	\text{Real Wealth} = \frac{\text{Nominal Wealth}}{\text{Price Index}}
\]

Nominal wealth is the wealth in dollars. Real wealth is the amount of goods and services one actually can buy.

\subsection{Who is Unaffected or Helped by Inflation?}

\paragraph{Flexible--Income Receivers} They derive their income only from social security, which is indexed with the CPI; so the real value they receive is the same.

\paragraph{Debtors} 

\subsection{Two Terms: Inflation}

\paragraph{Core Inflation} Underlying increases in the CPI after volatile food and energy prices are removed.

\paragraph{Hyperinflation} 
The purchasing power of money has decreased a lot.

\newpage
\section{Examination: Review}


\begin{enumerate}[label = \textbullet]
	\item Chapters 1, 2, 3, 4, 24, 26.

	\item Wall Street journal article.

	\item Basic calculator.
\end{enumerate}

\subsection*{Format}

\begin{enumerate}[label = \textbullet]
	\item Identification: this is when I'll give you a word, term, law and define as concise as possible.

	\item Short essay: answer with about five sentences. Some of them you can get bullet points for, with explanations. (Wall Street journal).

	\item Math and Graph: a problem to solve. questions about a graph.
\end{enumerate}

