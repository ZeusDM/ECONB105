\lecture{13}{April 09, 2021}{}

\subsection{Banking System as a Whole}

\paragraph{Assumptions/Simplifications}\

\begin{enumerate}[label = (\alph*)]
	\item Reserve Ratio = 20\%
	\item All banks are exactly meeting it: No excess reserves.
	\item Any acquired excess reserves are loaned out.
\end{enumerate}

\paragraph{Multiple Lending Process}

\begin{enumerate}
	\item[\textbf{A1.}] A deposit of \$100 is made to bank A.

		Required reserves are \$20.

	\item[\textbf{A2.}] Loans increase by \$80 and Demand Deposits increase \$80.

	\item[\textbf{A3.}] The borrower draws check for \$80 and Demand deposits decrease by \$80.
	
	\item[\textbf{B1.}] The check \$80 is deposited to bank B.

		Required reserves are \$16.

	\item[\textbf{B2.}] Loans increase by \$64 and Demand Deposits increase \$64.

	\item[\textbf{B3.}] The borrower draws chech for \$64 and Demand deposits decrease by \$64.
\end{enumerate}

This process will continue and the \$80 worth of Excess Reserves (initially) will result in the Banking System lending or creating \$400. So, the Banking system will lend by a multiple of $5$. $\$400 = \$80 \cdot 5$.

\paragraph{Monetary Multiplier}\ 

\[
	m = \text{Monetary Multiplier} = \frac{1}{\text{Required Reserved Ratio}} = \frac{1}{R}.
\]

To determine the amount of Maximum Demand Deposit money created or lended is determined by \[
	D = E \cdot m.
\]
