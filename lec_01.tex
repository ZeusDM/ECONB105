\newpage
\section{Chapter 1: Limits, Alternatives, and Choices}
\lecture{2}{February 16, 2021}{Ch. 1 (1/2)}
\subsection{Historical Background}

\paragraph{Adam Smith, philosopher} Smith's magnum opus is the book \emph{The Wealth of the Nations} (1776). Talks about an invisible hand (self-interest) which operates the economy: If each individual pursues its own interest, they frequently promote society's interest effectively.	Smith begins the tradition of classical economics.

\paragraph{Ricardo, Malthus, Mill} Start of 19\textsuperscript{th} century. David Ricardo writes about the theory of capital. Thomas Robert Malthus writes about labor and population theories. James Mill synthesizes ideas of economics.

\paragraph{Karl Marx} Karl Marx challenges the capitalist notion of common good, in the book \emph{Das Kapital} (1867). Capitalism exploits and will result in a revolution.

\paragraph{John Keynes} During the Great Depression, John Maynard Keynes writes \emph{The General Theory of Employment, Interest and Money} (1936). He attacks macro aspects of classical econonomy and the ``hands-off'' approach (free market). He advocates for government intervention to solve major problems in economics, such as employment and inflation. This work gives rise to Keynesian economics.

\paragraph{Neoclassical Economics} Second half of 20\textsuperscript{th} century. Neoclassical economics: rejection of Keynesian economics.

\subsection{Economic Terms}

\paragraph{Definition of economics}
Economics is a social science concerned with using scarce resources to obtain the maximum satisfaction of the unlimited human wants of society.

\paragraph{Ceteris paribus}
Other things being equal. In economics, it is usual to consider all variables are held constant, except the ones under consideration.

\paragraph{Correlation and causation}
Correlation is a systematics and dependable association between two sets of data.
It is not definitive.
Causation is definitive. \emph{``Correlation does not imply causation''}.

\subsection{Macroeconomics versus Microeconomics}

\paragraph{Definition of macroeconomics} Macroeconomics is concerned with aggregates (basic subdivisions, such as government, households, business sectors). No attention to specific units.
Examples of topics are: total outputs, total incomes.

\paragraph{Definition of microeconomics}
Microeconomics is concerned with specific economic units. Takes apart the aggregates.
Examples of topics are: price of a specific product, the income of a particular firm/household/industry. 

\subsubsection{Macroeconomics}

\paragraph{Fallacy of composition}
Generalizations made at the micro level may not be valid at the macro level.

\paragraph{Economic goals}
The consumer price index, gross domestic product are example of indexes economists uses to measure aspects of the economy.
Price stability and growth, measured by the indices above,  are goals.
Full employemnt is another goal: no workers should be involutarily out of work.
Balance of trade is another goal: there is a reasonable balance between exports and imports.

\subsubsection{Microeconomics}

\paragraph{Factors of production (economic resources)}
Land ––- all natural resourses which are usable of production. Capital ---  man-made resources used to produced goods and services; capital goods do not direcly satisfy human goods. Labor --- physical and mental human effort used to produce goods and services. Entrepreneurial ability --- combines labor, land and capital and produces products, makes non-routine decisions, inovates, bears risk.

These resources are limited.

\subsubsection{General}

\paragraph{The economizing problem} There are scarce resources, but unlimited human wants. Therefore, economic units search for an efficient allocation of resources.

\paragraph{Production possibilities model}
We will create a production possibilities model for the classroom. We will assume some things:
\begin{enumerate}[label = \textbullet]
	\item Efficiency: full employment and full capacity (of all economic resources).
	\item Fixed and limited resources --- may be reallocated.
	\item Fixed technology.
	\item We'll work in a economy that produces only two products: consumer good (satisfies imetiate need, ex. bread), capital good (satisfies more needs in the future, ex. robots).
\end{enumerate}

\paragraph{Production possisbilities table} It deals with the question of choice: how much of each good should we produce?

\begin{center}
\begin{tabular}{rccccc}
					 & $A$  & $B$ & $C$ & $D$ & $E$ \\
	Bread production & $0$  & $1$ & $2$ & $3$ & $4$ \\
	Robot production & $10$ & $9$ & $7$ & $4$ & $0$
\end{tabular}
\end{center}

The extremes ($A$ and $E$) are unrealistic. Society wants a combination of consumer and capital goods.

\paragraph{Production possibilities curve} \ 


\begin{figure}[ht]
	\centering
	\incfig[.45]{breadvsrobots}
	\caption{Example of production possibilities curve.}
\end{figure}

% Insert curve here

% IMPLICATION # 1
\paragraph{Opportunity Cost} The amount of other products that must be sacrificed to produce a unit of a given product.

\paragraph{Law of Increasing Opportinity Cost} As the amount of a product is increased, the amount of opportinity cost to produce a unit of this given product increases. Resources are better suited for some types of production than others. Thus, each time, to produce more breads, productivity is going to be lost.

\begin{center}
	\begin{tabular}{rcc}
							 & Robots & Bread \\
		Move from $A$ to $B$ & $-1$   & $+1$  \\
		Move from $B$ to $C$ & $-2$   & $+1$  \\
		Move from $C$ to $D$ & $-3$   & $+1$  \\
		Move from $D$ to $E$ & $-4$   & $+1$  \\
	\end{tabular}
\end{center}
