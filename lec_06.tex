\subsection{Two sides of GDP}
\lecture{6}{March 09, 2021}{Ch. 24 (2/2), Ch. 26 (1/2)}

\begin{enumerate}
	\item Introduction
	\begin{enumerate}
		\item Expenditures Approach: You're summing all the expenditures mades for final goods and services to measure the GDP.

		\item Income Approach: Summing all the incomes generated by the production of final goods and services to measure GDP.
	\end{enumerate}

\item Expenditures Approach

	\[
		\text{GDP} = C + I + G + X_n,
	\]
	where $C$ stands for consumption, $I$ stands for \emph{gross private domestic investment}; which can be divided in plant (construction, residential construction), equipment (computer systems, tools, etc), changes in inventory; $G$ stands for government (local, state or federal spending).

\item Income Approach

	\[
		\text{GDP} = W + \text{Rent} + I + \text{Profits} + T + \text{3 adjustments}
	\]

\item 3 adjustments

	Net foreign factor income: 

	Statistical discrepancy: it used to match expenditures and income approach.

	Depreciation or Consumption of Fixed Capital:

\item Expenditures = Income.
\end{enumerate}

\subsection{Other national Accounts}

\begin{enumerate}
	\item Net Domestic Product:
		\[
			\text{GDP} - \text{depreciation} = \text{NDP}.
		\]

	\item National Income: Income earned by Americans.
		\[
			\text{National Income} = \text{NDP} - \text{statistical discrepancy} + \text{net foreign factor income}.
		\]

	\item Personal Income: Income recieved. Tells the amount of income received for a year of production.
		\[
			\text{Personal Income} = \text{National Income} - \text{social security} - \text{taxes on production and imports} - \text{corporate income taxes} - \text{undistributed corporate profits} + \text{income received} 
		\]

	\item Disposable Income: the amount of income which househodlds can dispose if they choose to:
		 \[
			DI = PI - personal taxes
		\]

		\[
			DI = C + S
		\]

		S = savings

		SEE TABLE 7-4

\end{enumerate}

\newpage
\section{Chapter 26}

\subsection{The Business Cycle}

\paragraph{Define Business Cycle} Recurrent ups and downs over a period of years in the level of economic activity

INSERT FIGURE 8-2

\paragraph{Four stages of Business Cycle}

\begin{enumerate}
	\item Peak: full employment, full capacity.

	\item Recession: output and employment on decine. Prices are headed in a downward direction.

	\item Through: employment and output are at their lowest levels.

	\item Recovery: employment and output move toward full employment.

\end{enumerate}

\paragraph{Sources of Shocks} Why are $C$, $I$, $G$ or $X_n$ constantly changing?

\begin{enumerate}
	\item Irregular inovation.

	\item Productivity changes.

	\item Monetary factors.

	\item Political events.

	\item Financial instability.

\end{enumerate}

\subsection{Unemployment}

\paragraph{Definition} 
\begin{enumerate}
	\item Unemployment is the failure of the economy to fully employ the labor force.


	\item What is full employment? Something close to 100\% of the labor force, due to frictional and structural unemployment

	\item Natural Rate of Unemployment (NRU): When the economy is said to be producing its potential output. Given its resources, its producing as much as it can.
	
\end{enumerate}

\paragraph{Types of Unemployment}

\begin{enumerate}
	\item Frictional:
		Caused by workers voluntarily changing jobs, and temporary lay-offs. It is considered short-term. It is looked as desirable and inevitable.

	\item Structural:
		Caused by changes in the structure of demand for consumer goods and in technology. It is considered to be long-term.

	\item Cyclical:
		Caused by insufficient aggregate expenditures. As business activity increases, unenmployment is going to decrease. 
\end{enumerate}

\paragraph{Measuring Unemployment}

\begin{enumerate}
	\item What is the unemployment rate?

	\item Labor force is all people able and willing to work. Includes those employed, including part-time, and people willing to work.
\end{enumerate}

\subsection{Inflation}
\paragraph{Definition} Inflation is a rise in the average level of prices in the economy.

\paragraph{Measuring inflation}
\[
	\text{Rate of Inflation} = \frac{\text{Price Index}(\text{Current Year}) - \text{Price Index}(\text{Previous Year})}{\text{Price Index}(\text{Previous Year})}
\]
