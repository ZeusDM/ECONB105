\lecture{12}{April 06, 2021}{Ch. 31 (2/2), Ch. 32 (1/2)}

\subsection{What ``Backs'' the Money Supply?}

In other words, why are currency and transactions deposits considered money?

\paragraph{Money as a Debt} Realize money is a debt. 

\paragraph{Value of Money} Since money has no intrinsic value and is not backed by gold or another metal, then why is it considered money?

\begin{enumerate}[label = \textbullet]
	\item \textbf{Acceptability.} 
	\item \textbf{Legal Tender.} \emph{``Acceptability from the law.''} Citizens must accept it as a medium of exchange. Paper money is called \emph{fiat money}, which is anything that is money because the money decretes it to be.
	\item \textbf{Relative Scarcity.} Money derives its value from its scarcity relative to its usefulness. Usefulness refers to what the money can buy. Scarcity says that given a constant demand for money, its value or its purchasing power is determined by the supplied money.
\end{enumerate}

\paragraph{Money and Prices}

\begin{enumerate}[label = \textbullet]
	\item \textbf{Purchasing power of money varies inversely with the price level.} For example, when CPI increases, the purchasing power of a dollar decreases.
	\item \textbf{Historical Examples.} Germany after World War I. Argentina in the 1950s. Peru, Brazil...
	\item \textbf{Inflation can affect the acceptability of money as a medium of exchange.}
\end{enumerate}

\subsection{Determinants of Demand for Money}

Why public wants to hold $M_1$ money?

\begin{enumerate}[label = \textbullet] 
	\item \textbf{Transactions Demand ($D_t$).} Defined as the amount of $M_1$ money people want to hold to use for buying goods and services.
		Nominal GDP is the basic detarminant of $D_t$ (direct relationship).
	\item \textbf{Asset Demand ($D_a$).} Defined as the amount of $M_1$ money people want to hold as a store of value.
		It is the amount of their financial assets that they with to hold in the form of $M_1$ money.
		The lower the interest rate, the greater the $D_a$ for $M_1$ money; and vice versa. 
\end{enumerate}

\subsection{Structure of the U.S. Financial System}

\paragraph{Federal Reserve System: Central Banking} It has several levels:
\begin{enumerate}[label = \textbullet]
	\item \textbf{Board of Governors.} Seven members appointed by the President. Their duty is to run monetary policies.
	\item \textbf{Next level down under the Board.} The twelve regional/district federal reserve banks. Dispersed throughout the U.S..
	\item \textbf{Federal Open Market Committee.} Consist of the seven members of the Board of Governors plus five of the presidents of federal reserve banks (the NY FED president is always there). Sets policies toward open market operations (buying and selling of government bonds). 
\end{enumerate}

\paragraph{Commercial Banks}

\begin{enumerate}[label = \textbullet]
	\item \textbf{6,700 Commercial Banks in U.S.} See figure 17.2.

	\item A \emph{bank charter} is an official document permitting a banking company to commence business as a bank.
\end{enumerate}

\section{Chapter 32: Money Creation}

\subsection{Single Commercial Bank}

\paragraph{Introduction}

\begin{enumerate}[label = \textbullet]
	\item \textbf{Balance Sheet.}

		Assets are things that the bank owns; while liabilities are things that the bank owes to others.

		\begin{table}[ht]
			\centering
			\begin{tabular}{r|l}\toprule
				\textsc{Assets} $=$		& \textsc{Liabilities and Equity}	\\ \midrule
				Cash					& Demand Deposits					\\
				Deposits at the FED		& Savings Deposits					\\
				Short-term securities	& Certificate of Deposit (CD)		\\
				Loans					& Borrowings from the FED			\\
										& Capital (Equity)					\\ \bottomrule
			\end{tabular}
		\end{table}
\end{enumerate}

\newpage
\paragraph{Formation of a Commercial Bank}

\begin{enumerate}[label = \textbf{(\arabic*)}]
	\item[] Citizens in Wahoo decide need another bank. They sell $\$250$ thousand dollars on stock, and receive in cash.
	\item \textbf{Creating a bank:} Wahoo Bank
		
		\begin{table}[ht]
			\centering
			\begin{tabular}{r|l}\toprule
				\textsc{Assets}			& \textsc{Liabilities and Equity}	\\ \midrule
				Cash (\$250) & Capital Stock (\$250) \\ \bottomrule
			\end{tabular}
		\end{table}

	\item \textbf{Commence Operations.} The bank buys property and equipment.
		
		\begin{table}[ht]
			\centering
			\begin{tabular}{r|l}\toprule
				\textsc{Assets}			& \textsc{Liabilities and Equity}	\\ \midrule
				Cash (\$10) & Capital Stock (\$250)\\ 
				Property and Equipment (\$240) & \\ \bottomrule
			\end{tabular}
		\end{table}

	\item \textbf{Accepts Deposits.} A business deposits $\$100$ thousand dollars.
		
		\begin{table}[ht]
			\centering
			\begin{tabular}{r|l}\toprule
				\textsc{Assets}			& \textsc{Liabilities and Equity}	\\ \midrule
				Cash (\$110) & Capital Stock (\$250) \\ 
				Property and Equipment (\$240) & Demand Deposits (\$100) \\ \bottomrule
			\end{tabular}
		\end{table}

	\item[] Business of Wahoo deposits $\$100,000$.
	\item \textbf{Depositing Reserves in a Federal Reserve Bank.}
		Required Reserves are the minimum amount which a bank must keep on deposits with a FED or in the form of vault cash.

		\begin{enumerate}[label = (\alph*)]
			\item Amount of Required Reserves = \% of total demand deposits.

			\item This percentage is called the Reserve Ratio $= \dfrac{\text{Commercial Bank's Required Reserves}}{\text{Commercial Bank's Total Demand Deposits}}$.
				(Fractional Reserve System of Banking: The required reserves are less than $100\%$ of the total demand deposits.)

		%\begin{table}[ht]
		\begin{center}
			\vspace{1em}
			\begin{tabular}{r|l}\toprule
				\textsc{Assets}			& \textsc{Liabilities and Equity}	\\ \midrule
				Cash (\$0)				& Capital Stock (\$250)				\\ 
				Reserves (\$110)		& Demand Deposits (\$100)			\\ 
				Property (\$240)		& 									\\ \bottomrule
			\end{tabular}
			\vspace{1em}
		\end{center}
		%\end{table}

			\item If the Reserve Ratio is $20\%$, their required reserves is $\$20$ thousand dollars.

				The bank thinks Demand Deposits increase so the bank sends an extra $\$90$ thousand dollars to FED.

				Now, the total reserves are $\$110,000$.
	
			\item[\textbullet] \textbf{Definition of Excess Reserves:} Amount by which total reserves exceed required reserves.
				
				\[
					\text{Total Reserves} - \text{Required Reserves} = \text{Excess Reserves}
				\]

				\[
					\text{Total Reserves} = \text{Vault Cash} + \text{Deposits at the FED}
				\]

			\item[\textbullet] The purpose of Required Reserves is the means by which the Board of Governors can control the lending ability of banks; therefore avoiding bank failures.

		\end{enumerate}

	\item Check is drawn against Wahoo Bank:

		\begin{enumerate}[label = (\alph*)]
			\item A major depositor of Wahoo bank writes a check against his deposits for \$50 thousand dollars for farm machinery.

			\item A major depositor gives check to farm machinery company and is deposited in their bank: Bank X.

				\begin{enumerate}[label = (\roman*)]
					\item How will Bank X collect claim of \$50 thousand dollars?

					\item Check is sent to district bank where:

						 \begin{enumerate}[label = (\alph*)]
							 \item clerk increases Bank X's reseves by \$50 thousand dollars and
							 \item decreases Wahoo's bank reserves by \$50 thousand dollars.
						\end{enumerate}

					\item Same check then is ``cleared'' and sent to Wahoo bank:
						
						\begin{enumerate}[label = (\alph*)]
							\item Now, Demand Deposits decrease by \$50 thousand dollars and
							\item Reserves of Wahoo bank decrease by \$50 thousand dollars.
						\end{enumerate}
				\end{enumerate}
		\end{enumerate}
		
		%\begin{table}[ht]
		\begin{center}
			\vspace{1em}
			\begin{tabular}{r|l}\toprule
				\textsc{Assets}			& \textsc{Liabilities and Equity}	\\ \midrule
				Cash (\$0)				& Capital Stock (\$250)				\\ 
				Reserves (\$60)			& Demand Deposits (\$50)			\\ 
				Property (\$240)		& 									\\ \bottomrule
			\end{tabular}
			\vspace{1em}
		\end{center}
		%\end{table}

\end{enumerate}


