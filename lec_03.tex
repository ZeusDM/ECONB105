\lecture{3}{February 23, 2021}{Ch. 3 (2/3)}

\subsection{Demand}
\paragraph{Definition} The relationship between the price of a product and the quantity people will buy.
\paragraph{Law of Demand} Ceteris paribus, as price increases, the corresponding quantity demanded is going to decrease. Negative correlation.
Ceteris paribus, quantity demanded is a function of price. See \cref{fig:demand_individual_curve}.

\begin{figure}[ht]
	\centering
	\incfig[.4]{demand_individual_curve}
	\caption{Demand curve}
	\label{fig:demand_individual_curve}
\end{figure}

\paragraph{Individual vs. Market Demand} See \cref{fig:individual_demand_to_market}.
\begin{figure}[ht]
	\centering
	\makebox[\textwidth][c]{
		\incfig[1.1]{individual_demand_to_market}
	}
	\caption{Market demand curve from individual demand curve}
	\label{fig:individual_demand_to_market}
\end{figure}

\paragraph{Changes in Quantity Demanded} Quantity demanded changes because price changes. Graphically, it is a movement along a fixed demand curve caused by only a change in price. See \cref{fig:changes_in_quantity_demanded}

\begin{figure}[ht]
	\centering
	\incfig[.4]{changes_in_quantity_demanded}
	\caption{Changes in quantity demanded}
	\label{fig:changes_in_quantity_demanded}
\end{figure}

\paragraph{Determinants of Demand}
\begin{enumerate}[label = \textbullet]
	\item Consumer Tastes/Preferences
	\item Number of Buyers (positive correlation)
	\item Income: normal goods --- demand varies direcly with income; inferior goods --- demand varies inversely with income.
	\item Prices of Related Goods: price of substitute goods (positive correlation), price of complementary goods (negative correlation).
	\item Consumer Expectations: future prices, future availability and future income.
\end{enumerate}
\paragraph{Changes in Demand} Shift of the whole demand curve because of determinants above (except from price). See \cref{fig:changes_in_demand}.

\begin{figure}[ht]
	\centering
	\incfig[.4]{changes_in_demand}
	\caption{Changes in demand}
	\label{fig:changes_in_demand}
\end{figure}

\subsection{Supply}
\paragraph{Definition} The relationship between the price of a product and the quantity people will sell.
\paragraph{Law of Supply} Ceteris paribus, as price increases, the corresponding quantity supplied will increase. Positive correlation. Ceteris paribus, quantity supplied is a function of price. See \cref{fig:supply_individual_curve}.

\begin{figure}[ht]
	\centering
	\incfig[.4]{supply_individual_curve}
	\caption{Supply curve}
	\label{fig:supply_individual_curve}
\end{figure}

\paragraph{Individual vs. Market Supply}
Calculated the same as demand.

\paragraph{Changes in Quantity Supplied} Quantity supplied changes because price changes. Graphically, it is a movement along a fixed supply curve caused by only a change in price. See \cref{fig:changes_in_quantity_supplied}.
\begin{figure}[ht]
	\centering
	\incfig[.4]{changes_in_quantity_supplied}
	\caption{Changes in quantity supplied}
	\label{fig:changes_in_quantity_supplied}
\end{figure}


